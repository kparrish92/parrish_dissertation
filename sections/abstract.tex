The present dissertation examined the initial state of L3 phonology in naive speakers of French and German who were Spanish-English bilinguals.
The present work analyzed perception (Chapter 3), production (Chapter 4) and the perception production interface (Chapter 5).
The results reavealed that these bilinguals are influenced by both of their known languages in both perception and production.
Additionally, there is evidence that L3 perception and production are differently impacted by their known languages. The results provide important counterevidence for L3 models such as the Typolgocial Primacy Model (Rothman, 2010) or the L2 Status Factor (Bardel \& Falk, 2007), which predict that whole-language influence occurs. The present work is best explained by the Linguistic Proximity Model (Westergaard et al., 2017), but was not necessarily predicted by it.
As a result, the present work serves as an example of the necessity for a future model of L3 phonology, which takes into account language typology, order of acquisition, segmental acoustics and explores future directions to explain the variability observed in the present work. 

